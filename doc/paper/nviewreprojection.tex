\chapter{N/Multiview Reprojection}

\section{Overview}
At this stage we are exclusivley in $\mathbb{R}^{3}$.
For the purpose of our software, we expect points in the form $P = (p_x, p_y, p_z)$
and their corresponding orientation as a unit vector $\hat{U} = (\hat{u}_x, \hat{u}_y, \hat{u}_z)$,
we expect these together in a tuple $ ((p_x, p_y, p_z), (\hat{u}_x, \hat{u}_y, \hat{u}_z))_n $.
This tuple comes with several ($n$ many) tuples which all uniquely correspond with a matched set.
So we have a $\mathbb{R}^{3}$ match set $M = \{ (P, \hat{U})_0, (P, \hat{U})_1, ... , (P, \hat{U})_n \}$
where $n$ is the numnber of tuple pairs and has a one-to-one correspondence with
the number of views in which the $\mathbb{R}^{2}$ match was found.
We all also get a set of $\mathbb{R}^{3}$ matches,
$M_i$,
resulting in a set which has a one-to-one correspondence with the total number
of points we should have after reprojection.

\begin{center}
\begin{tikzpicture}

% the grid and stuff
\draw[step=1cm,gray,very thin] (0,0) grid (5,5);
\foreach \x in {0,1,2,3,4,5}
   \draw (\x cm,1pt) -- (\x cm,-1pt) node[anchor=north] {$\x$};
\foreach \y in {0,1,2,3,4,5}
    \draw (1pt,\y cm) -- (-1pt,\y cm) node[anchor=east] {$\y$};

% the arrows
\draw[->](1,0.5) -- (1.5,1.5);
\draw[->](2.5,0.5) -- (2.5,1.5);
\draw[->](4,0.5) -- (3.4,1.5);

% dotted lines
\draw[red,thin,dashed] (1.5,1.5) -- (3.25,5.0);
\draw[red,thin,dashed] (2.5,1.5) -- (2.5,5.0);
\draw[red,thin,dashed] (3.4,1.5) -- (1.3,5.0);

% points
\filldraw (1,0.5) circle[radius=1.5pt];
\node[left=8pt of {(1,0.5)}, outer sep=1pt] {$(P,\hat{U})_0$};
\filldraw (2.5,0.5) circle[radius=1.5pt];
\node[below=8pt of {(2.5,0.5)}] {$(P,\hat{U})_1$};
\filldraw (4,0.5) circle[radius=1.5pt];
\node[right=8pt of {(4,0.5)}, outer sep=1pt] {$(P,\hat{U})_2$};
\filldraw (2.45,3.25) circle[radius=1.5pt];
\node[right=8pt of {(2.45,3.25)}, outer sep=1pt] {$C$};

% labels

\end{tikzpicture}
\end{center}

I know I said we're in $\mathbb{R}^{3}$ (and we are), just imagine that the diagram
above illustrates the values within a particular Z plane. Note that the diagram
could be similar if any arbitary plane is chosen.\\
\\
In the above diagram, assume that point $C$ is the correct real world point and
has no orientation. We're trying to make a best guess at was $C$ is given our
imperfect information.

\section{3 by 3 Inversion Method}
Caleb (me the person writing this) made up the name to this method, so don't
search it because you probably won't find anything about it. \\
\\
First, remember the identity:
\[
(m \times n) \cdot (m \times n) = || m ||^2  || n ||^2 - (m \cdot n)^2
\]
Then note, we have the distance function, measuring how much a given
$((p_x, p_y, p_z), (\hat{u}_x, \hat{u}_y, \hat{u}_z))_n$ tuple (which represents
a line) misses the real world targe point $C$. Not this function is calculated
for each tuple.
\[
D_n = \frac{||(C - P_n) \times \hat{U}_n || }{ || \hat{U}_n ||}
\]
We will want to use the square of the distance (as is common in many optimization
problems) to insure convex optimization and positive distance values. We also use
the identity mentioned above to get our primary distance equation. Then, taking
the first derivative of the distance function will, to find a $0$ solution, will
give us a local minimum value.
\begin{align*}
    D_n &= \frac{||(C - P_n) \times \hat{U}_n || }{ || \hat{U}_n ||} \\
    D_n^2 &= \Big( \frac{||(C - P_n) \times \hat{U}_n || }{ || \hat{U}_n ||} \Big)^2 \\
    D_n^2 &= \frac{||(C - P_n) \times \hat{U}_n ||^2  }{ || \hat{U}_n ||^2 }\\
    D_n^2 &= \frac{ || C - P_n ||^2   || \hat{U}_n ||^2 - ||(C - P_n) \cdot \hat{U}_n ||^2 }{ || \hat{U}_n ||^2 }\\
    D_n^2 &= || C - P_n ||^2  - \frac{ ||(C - P_n) \cdot \hat{U}_n ||^2 }{|| \hat{U}_n ||^2} \\
    \frac{d D_n^2}{d C} &= 2 (C - P_n) - 2 \hat{U}_n  \frac{ (C - P_n) \cdot \hat{U}_n }{ || \hat{U}_n ||^2 }
\end{align*}
So, we need to find a zero for the following (note that we are dealing with a
vector in $\mathbb{R}^{3}$, so $0 = [0 , 0, 0]^T$). the value $m$ is the total number
of $\mathbb{R}^{3}$ match points:

\begin{align*}
  0 &= \sum_{n=0}^m C - P_n - \hat{U}_n \frac{ (C - P_n) \cdot \hat{U}_n }{ || \hat{U}_n ||^2 }\\
  0 &= \sum_{n=0}^m C - P_n - \frac{ \hat{U}_n ( C \cdot \hat{U}_n ) }{ || \hat{U}_n ||^2 } + \frac{ \hat{U}_n ( P_n \cdot \hat{U}_n ) }{ || \hat{U}_n ||^2 } \\
  0 &= \sum_{n=0}^m C - P_n - \frac{ \hat{U}_n \hat{U}_n^T C }{ || \hat{U}_n ||^2 } + \frac{ \hat{U}_n \hat{U}_n^T P_n }{ || \hat{U}_n ||^2 }\\
  0 &= \sum_{n=0}^m \Big( I -  \frac{ \hat{U}_n \hat{U}_n^T }{ || \hat{U}_n ||^2 }  \Big) C - \Big( P_n - \frac{ \hat{U}_n \hat{U}_n^T P_n }{ || \hat{U}_n ||^2 } \Big)
\end{align*}

Notice this is of the form $Ax = b$ because we now have $0 = Ax - b$. The next
thing to note is we can remove the summations and get a system that results in
taking an inverse of a 3 by 3 matrix.\\
\\
Notice the possible expansion:
\begin{align*}
  0 &= \sum_{n=0}^m \Big( I -  \frac{ \hat{U}_n \hat{U}_n^T }{ || \hat{U}_n ||^2 }  \Big) C - \Big( P_n - \frac{ \hat{U}_n \hat{U}_n^T P_n }{ || \hat{U}_n ||^2 } \Big)\\
  0 &= \Bigg( \Big( I -  \frac{ \hat{U}_0 \hat{U}_0^T }{ || \hat{U}_0 ||^2 }  \Big) + \Big( I -  \frac{ \hat{U}_1 \hat{U}_1^T }{ || \hat{U}_1 ||^2 } \Big) + \Big( I -  \frac{ \hat{U}_2 \hat{U}_2^T }{ || \hat{U}_2 ||^2 } \Big) + ...  \Bigg) C - \Bigg( \Big( P_0 - \frac{ \hat{U}_0 \hat{U}_0^T P_0 }{ || \hat{U}_0 ||^2 } \Big) + \Big( P_1 - \frac{ \hat{U}_1 \hat{U}_1^T P_1 }{ || \hat{U}_1 ||^2 } \Big) + ... \Bigg) \\
  0 &= (A) C - (b)\\
  AC &= b \\
  C &= A^{-1}b
\end{align*}
So, the meat of this method is to calculate the $A$ matrix's inverse and muplyply
it by vector $b$ to find the estimated point $C$. succinctly, these are calcualted:
\[
A = \sum_{n=0}^m \Big( I -  \frac{ \hat{U}_n \hat{U}_n^T }{ || \hat{U}_n ||^2 } \Big)
\]
\[
b = \sum_{n=0}^m \Big( P_n - \frac{ \hat{U}_n \hat{U}_n^T P_n }{ || \hat{U}_n ||^2 } \Big)
\]

























% yeet
